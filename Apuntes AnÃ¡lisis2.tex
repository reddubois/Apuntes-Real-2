\documentclass{article}
\usepackage[utf8]{inputenc}
\usepackage[spanish,es-nodecimaldot]{babel}
\usepackage{amsmath}
\usepackage{amssymb}
\usepackage{amsthm}
\usepackage{mathtools}
\usepackage[bottom]{footmisc}
\usepackage{xcolor}
\DeclarePairedDelimiter{\ceil}{\lceil}{\rceil}
\usepackage{graphicx}
\renewcommand\qedsymbol{$\blacksquare$}
\usepackage{enumitem}
\newcommand*{\vcenteredhbox}[1]{\begin{tabular}{@{}c@{}}#1\end{tabular}}
\newcommand*{\colorboxed}{}
\usepackage{physics}

\DeclareMathOperator{\interior}{int}
\DeclareMathOperator{\osc}{osc}

\def\colorboxed#1#{%
  \colorboxedAux{#1}%
}
\newcommand*{\colorboxedAux}[3]{%
  % #1: optional argument for color model
  % #2: color specification
  % #3: formula
  \begingroup
    \colorlet{cb@saved}{.}%
    \color#1{#2}%
    \boxed{%
      \color{cb@saved}%
      #3%
    }%
  \endgroup
}

\usepackage{fancyhdr}
\usepackage[left=1.2in,right=1.2in,top=1in,bottom=1.25in,%
            footskip=.25in]{geometry}
\setlength\parindent{0pt}


\title{Análisis Real 2: Apuntes de Clase}
\author{\Large Rafael Dubois\\ Universidad del Valle de Guatemala \\ \texttt{dub19093@uvg.edu.gt}}
\date{\today}

\pagestyle{fancy}
\fancyhf{}
\renewcommand{\headrulewidth}{2pt}
\fancyfoot{}
\rhead{\footnotesize Análisis de variable real 2}
\lhead{\footnotesize Apuntes de Análisis}
\rfoot{\thepage}
\lfoot{}
\setlength{\headheight}{28pt}

\begin{document}

\maketitle
\lhead{\footnotesize Universidad del Valle de Guatemala \\ 
\footnotesize Departamento de Matemática \\ 
\footnotesize Análisis de Variable Real 2}
\rhead{\footnotesize Licenciatura en Matemática Aplicada \\ 
\footnotesize Rafael Dubois \\ 
\footnotesize Carné 19093}
\thispagestyle{fancy}

\section*{Requerimientos para la integral de Riemann según Darboux}

\textbf{Nota:} La teoría de estas integrales se refiere a funciones acotadas (salvo se diga lo contrario).

\subsection*{Partición de un intervalo}

Una partición del intervalo $[a,b]$ es un conjunto finito $P=\{x_0,x_1,\ldots,x_n\}$ tal que 
$$a=x_0<x_1<\cdots<x_n=b.$$
El conjunto de todas las posibles particiones de $[a,b]$ se denota por $P[a,b]$.

\subsubsection*{Refinamiento}
Una partición $P'\in P[a,b]$ es un refinamiento de $P\in P[a,b]$ si $P\subseteq P'$. Esto se denota por $P'\leq P$. Nótese que:
\begin{itemize}
\item Para todo $P\in P[a,b]$ se tiene que $P\leq P$.
\item $P'\leq P$ y $P\leq P'$ si y solo si $P=P'$.
\item Si $P'\leq P$ y $P''\leq P'$, entonces $P''\leq P$.
\end{itemize}
Por lo tanto, $\leq$ es una relación de orden parcial.

\subsubsection*{Longitud de un intervalo}
Para $P\in P[a,b]$, se denota $\Delta x_k=x_k-x_{k-1}$ a la longitud del $k$-ésimo subintervalo en la partición. Nótese que
$$\sum_{k=1}^{n}\Delta x_k=b-a.$$

\subsubsection*{Norma de un intervalo}
Sea $P\in P[a,b]$. Se define $\norm{P}=\max\{\Delta x_k: 1\leq k\leq n,\hspace{5pt} k\in\mathbb{N}\}$ como la norma (o malla) de $P$. Nótese que $P'\leq P$ implica que $\norm{P'}\leq\norm{P}$.

\newpage
\pagestyle{fancy}
\fancyhf{}
\renewcommand{\headrulewidth}{2pt}
\fancyfoot{}
\rhead{\footnotesize Análisis de variable real 2}
\lhead{\footnotesize Apuntes de Julio}
\rfoot{\thepage}
\lfoot{}
\setlength{\headheight}{28pt}

\subsection*{Suma superior e inferior de Darboux}
Sea $P\in P[a,b]$ y sean, para $1\leq k\leq n$ con $k\in\mathbb{N}$,
\begin{align*}
M_k(f)=\sup\{f(x): x\in[x_{k-1},x_k]\}, && m_k(f)=\inf\{f(x): x\in[x_{k-1},x_k]\}.
\end{align*}
Entonces, los números
\begin{align*}
U(P,f)=\sum_{k=1}^{n}M_k(f)\Delta x_{k}, && L(P,f)=\sum_{k=1}^{n}m_k(f)\Delta x_{k}
\end{align*}
se llaman suma superior y suma inferior de Darboux de $f$ para $P$, respectivamente.

\subsubsection*{Propiedades}
Sean $P$ y $P'$ en $P[a,b]$. Entonces,
$$P'\leq P\implies U(P',f)\leq U(P,f) \hspace{5pt} \text{ y }\hspace{5pt} L(P',f)\geq L(P,f).$$
Además, para $P$ y $P'$ en $P[a,b]$ se tiene también que
$$L(P,f)\leq U(P',f).$$
Y finalmente, también se cumple
$$U(P,f)-L(P,f)\geq U(P',f)-L(P',f)\geq 0.$$

\section*{Integral de Riemann según Darboux}

\subsection*{Definición}
Si existen las integrales superior e inferior de Riemann según Darboux, y estas son iguales, se define a la integral de Riemann como este valor. Si estas integrales difieren para alguna $f$, entonces dicha función no es Riemann-integrable.

\subsection*{Integral superior de Riemann según Darboux}
Se define la integral superior de Riemann de $f$ en $[a,b]$ como
$$\overline{\int_{a}^{b}}f=\inf\big\{U(P,f): P\in P[a,b]\big\}.$$

\subsection*{Integral inferior de Riemann según Darboux}
Se define la integral inferior de Riemann de $f$ en $[a,b]$ como
$$\underline{\int_{a}^{b}}f=\sup\big\{L(P,f): P\in P[a,b]\big\}.$$

\vspace{10pt}
\textbf{Nota: Aquí terminó la clase del 5 de julio del 2021.}
\newpage

\subsection*{Desigualdad de las integrales inferior-superior}

Para una función $f$ definida en $[a,b]$, siempre se tiene $\displaystyle \underline{\int_{a}^{b}}f\leq \overline{\int_{a}^{b}}f$.

\subsection*{Notación del conjunto de funciones Riemann-integrables}

Al conjunto de funciones Riemann-integrables sobre $[a,b]$ se le denota por $R[a,b]$.

\subsection*{Criterio de Cauchy/Riemann para integrabilidad}

Una función acotada $f:[a,b]\to\mathbb{R}$ es Riemann-integrable sobre $[a,b]$ si y solo si para todo $\varepsilon>0$ existe $P_{\varepsilon}\in P[a,b]$ tal que para todo $P\leq P_{\varepsilon}$ se tiene que $0\leq U(P,f)-L(P,f)\leq\varepsilon$.

\subsection*{Conservación de Riemann-integrabilidad}

Sea $f\in R[a,b]$ y $[c,d]\subseteq [a,b]$. Entonces, $f\in R[c,d]$.

\subsection*{Suma de Riemann}

Dada una partición $P$ de $n$ intervalos, una función $f$, y una muestra $t_1,t_2,\ldots,t_n$ para $t_k$ en el $k$-avo intervalo, se define la suma de Riemann como
$$S(P,f,\{t_k\})=\sum_{k=1}^n f(t_k)\Delta x_k.$$

\subsection*{Primer teorema de caracterización de funciones Riemann-integrables}

Una función $f$ es Riemann-integrable si y solo si existe un número $A$ con la propiedad de que para todo $\varepsilon>0$ existe $P_{\varepsilon}\in P[a,b]$ tal que $P_{\varepsilon}\leq P$ y para cualquier muestra $t_k\in[x_{k-1},x_k]\subset[a,b]$ con $1\leq k\leq n$ entero, se cumple
$$|S(P,f,\{t_k\})-A|<\varepsilon.$$
Como nota, este número $A$ se obtiene (informalmente) con:
$$A=\lim_{\norm{P}\to 0}\left(\sum_{k=1}^{n} f(t_k)\Delta x_k\right)=\lim_{n\to\infty}\left(\sum_{k=1}^{n} f(t_k)\Delta x_k\right).$$

\vspace{10pt}
\textbf{Nota: Aquí terminó la clase del 8 de julio del 2021.}
\newpage

\subsection*{Oscilación de una función}

La oscilación de una función acotada $f$ sobre un conjunto $A$ se define $\osc(f)=\sup(f)-\inf(f)$. De esta manera, dada $f:[a,b]\to\mathbb{R}$ acotada y $P\in P[a,b]$, entonces
$$U(P,f)-L(P,f)=\sum_{k=1}^{n}\osc_{I_k}(f)\Delta x_k.$$

\subsubsection*{Propiedad de la oscilación de dos funciones}

Sean $f,g:[a,b]\to\mathbb{R}$ funciones acotadas y suponga que $g\in R[a,b]$. Si existe $c>0$ tal que $\osc_I(f)\leq c\osc_I(g)$ sobre cada $I\subseteq[a,b]$, entonces $f\in R[a,b]$.

\subsection*{Segundo teorema de caracterización de funciones Riemann-integrables}

Una función $f$ es Riemann-integrable si y solo si existe una sucesión de particiones $(P_n)$ con $P_n\in P[a,b]$ para todo entero positivo $n$, tal que
$$\lim_{n\to\infty}\left[U(P_n,f)-L(P_n,f)\right]=0.$$
En este caso, $\displaystyle\int_a^b f=\lim_{n\to\infty}U(P_n,f)=\lim_{n\to\infty}L(P_n,f)$.

\subsection*{Teorema de Riemann-integrabilidad de funciones continuas}

Toda función continua $f:[a,b]\to\mathbb{R}$ es Riemann-integrable.

\subsection*{Teorema de Riemann-integrabilidad de funciones monótonas}

Toda función monótona $f:[a,b]\to\mathbb{R}$ sobre un compacto (la función es entonces acotada, sus únicas discontinuidades son de salto y la cantidad de estas es, como máximo, infinita contable) es Riemann-integrable.

\vspace{10pt}
\textbf{Nota: Aquí terminó la clase del 12 de julio del 2021.}
\newpage

\section*{Propiedades de la integral}

\subsection*{Multiplicación por una constante}

Para $f:[a,b]\to\mathbb{R}$ y $c\in\mathbb{R}$, se cumple $\displaystyle\int_a^b c\cdot f=c\int_a^b f$. 

\subsection*{Nota sobre supremos e ínfimos, integrabilidad de la suma}

Como recordatorio, se cumple para funciones $f$ y $g$ que:
\begin{align*}
\sup_I(f+g)&\leq \sup_I(f)+\sup_I(g),\\
\inf_I(f+g)&\geq \inf_I(f)+\inf_I(g).
\end{align*}
Por lo tanto,
$$\osc_I(f+g)\leq \osc_I(f)+\osc_I(g).$$

\subsection*{Cerradura de R(I) bajo la suma}

Si $f,g\in R(I)$, entonces $f+g\in R(I)$.

\subsection*{Linealidad de las integrales superiores e inferiores}

En general, las integrales superiores e inferiores de funciones no Riemann-integrables no necesariamente cumplen con linealidad. Esto no es cierto para funciones Riemann-integrables.

\subsection*{Definición de índices de las integrales}

Dado $f:[a,b]\to\mathbb{R}$, se define:
\begin{align*}
\overline{\int_a^b}f=-\overline{\int_b^a}f && \underline{\int_a^b}f=-\underline{\int_b^a}f.
\end{align*}
Si $a=b$, se tiene que ambas integrales son cero.

\subsection*{Desigualdad de las integrales superiores e inferiores}

\begin{align*}
\overline{\int_a^b}(f+g)&\leq \overline{\int_a^b}f+\overline{\int_a^b}g,\\
\underline{\int_a^b}(f+g)&\geq \underline{\int_a^b}f+\underline{\int_a^b}g.
\end{align*}

\subsection*{Linealidad de la integral}

En general, si $f,g\in R[a,b]$, $\displaystyle \int_a^b (f+g)=\int_a^b f+\int_a^b g$.

\subsection*{Monoticidad de la integral}

Si $f,g\in R[a,b]$ cumplen $f(x)\leq g(x)$ para todo $x\in[a,b]$, entonces $\displaystyle \int_a^b f\leq\int_a^b g$. No necesariamente se cumple el converso de esta implicación.

\subsection*{Lema de acotación de la integral}

Sea $f\in R[a,b]$, con $M=\sup(f)$ y $m=\inf(f)$. Entonces, $m(b-a)\leq\displaystyle\int_a^b f\leq M(b-a)$.

\subsection*{Teorema del valor medio de la integral}

Para toda $f:[a,b]\to\mathbb{R}$ función continua, existe $c\in[a,b]$ tal que $f(c)=\displaystyle\frac{1}{b-a}\int_a^b f$.

\vspace{10pt}
\textbf{Nota: Aquí terminó la clase del 15 de julio del 2021.}

\subsection*{Desigualdad triangular de la integral}

Sea $f\in R[a,b]$. Entonces, $|f|\in R[a,b]$. Además,
$$\left|\int_a^b f\right|\leq \int_a^b |f|.$$

\subsection*{Cerradura de R(I) bajo el producto}

Si $f,g\in R(I)$, entonces $f\cdot g\in R(I)$.

\subsection*{Acotaciones de las integrales superiores e inferiores}

Sea $f:[a,b]\to\mathbb{R}$ tal que $m\leq f(x)\leq M$ para todo $x\in[a,b]$. Luego,
\begin{align*}
m(b-a)\leq \underline{\int_a^b} f, && \overline{\int_a^b} f \leq M(b-a).
\end{align*}

\subsection*{Teorema cero del cálculo integral}

Sea $I$ un intervalo y $f:I\to\mathbb{R}$ una función acotada en $I$. Sea $a\in I$ y para todo $x\in I$ considere
\begin{align*}
\overline{F}(x)= \overline{\int_a^x} f, && \underline{F}(x)= \underline{\int_a^x} f.
\end{align*}
Entonces $\overline{F}$ y $\underline{F}$ son continuas en $I$ y además, si $f$ es continua en $c$, son diferenciables en $c$ de manera que $\overline{F}'(c)=\underline{F}'(c)=f(c)$.

\subsection*{Primer teorema fundamental del cálculo}

Sea $f\in R[a,b]$ y $F\in R[a,b]$ tal que para todo $x\in[a,b]$,
$$F(x):=\int_a^x f.$$
Si $f$ es continua en $c\in[a,b]$, entonces $F$ es diferenciable en $c$ y $F'(c)=f(c)$.

\subsection*{Segundo teorema fundamental del cálculo}

Sea $f\in R[a,b]$ y sea $G$ una función derivable en $(a,b)$ tal que $G'=f$. Entonces,
$$\int_a^b f=G(b)-G(a).$$

\vspace{10pt}
\textbf{Nota: Aquí terminó la clase del 19 de julio del 2021.}
\newpage

\subsection*{Teorema de integrabilidad dada la semi-igualdad}

Sean $f,g:[a,b]\to\mathbb{R}$ funciones acotadas y tales que $f(x)=g(x)$ en casi todo $[a,b]$ excepto en una cantidad finita de puntos en dicho intervalo. Entonces, $f\in R[a,b]$ si y solo si $g\in R[a,b]$, y sus integrales son iguales.

\subsection*{Límite en las fronteras de integración}

Suponga que $f:[a,b]\to\mathbb{R}$ es acotada e integrable en $[a,r]$ para cada $r\in(a,b)$. Entonces, $f\in R[a,b]$ y
$$\int_a^b f=\lim_{r\to b^{-}} \int_a^r f.$$

\vspace{10pt}
\textbf{Nota: Aquí terminó la clase del 22 de julio del 2021.}

\subsection*{Lema para la integral del producto}

Sean $f,g\in R[a,b]$. Para cada $P\in P[a,b]$, cada selección $\{t_k\}$ de $P$, y cualquier selección $\{f_k\}$ de $f_k\in\{M_k(f),m_k(f)\}$, considérese
$$\omega(P,f,g)=\sum_{k=1}^n f_k g(t_k)\Delta x_k.$$
Entonces, $\omega(P,f,g)$ converge en el sentido de Riemann al valor de $\displaystyle\int_a^b fg$.

\subsection*{Teorema de Bonnet}

Sea $f\in R[a,b]$ y $g:[a,b]\to\mathbb{R}$ una función no negativa, acotada y monótona decreciente. Entonces, existe $\mu\in[a,b]$ tal que
$$\int_a^b fg=g(a)\int_a^{\mu} f.$$

\subsection*{Teorema de integración por partes}

Sean $f,g\in C[a,b]$ diferenciables en $(a,b)$, y tales que $f',g'\in R[a,b]$. Luego,
$$\int_a^b fg'=f(b)g(b)-f(a)g(a)-\int_a^b f'g.$$

\subsection*{Teorema de integración por sustitución}

Suponga que $g:I\to\mathbb{R}$ es diferenciable y que $g'\in R(I)$, y sea $J=g(I)$. Si $f:J\to\mathbb{R}$, entonces para todo $a,b\in I$ se cumple
$$\int_a^b f(g(x))g'(x)\dd{x}=\int_{g(a)}^{g(b)}f(u)\dd{u}.$$

\vspace{10pt}
\textbf{Nota: Aquí terminó la clase del 26 de julio del 2021.}
\newpage

\subsection*{Propiedad de interés (requerirá prueba)}

Suponga que $f_n\in R[a,b]$ para todo $n\in\mathbb{Z}^{+}$ y suponga que $f_n\to f$ de manera uniforme sobre $[a,b]$. Entonces, $f\in R[a,b]$ y
$$\lim_{n\to\infty}\int_a^b f_n =\int_a^b f.$$

\section*{Integrales impropias}

\subsection*{Definición 1}

Suponga que $f:(a,b]\to\mathbb{R}$ es Riemann integrable en $[c,b]$ para $c\in(a,b)$. Entonces, la integral impropia de $f$ sobre $[a,b]$ es
$$\int_a^b f=\lim_{\varepsilon\to 0}\int_{a+\varepsilon}^b f.$$
La integral impropia converge si este límite existe. De lo contrario, la integral diverge.

\subsection*{Definición 2}

Suponga que $f:[a,\infty)\to\mathbb{R}$ es integrable sobre $[a,b]$ para $r\in(a,\infty)$. Entonces, la integral impropia de $f$ es
$$\int_a^{\infty} f=\lim_{r\to \infty}\int_{a}^r f.$$

\subsubsection*{Integral de Frullani}

Sea $f:[0,\infty)\to\mathbb{R}$ una función continua cuyo límite en el infinito existe. Luego,
$$\int_0^{\infty}\frac{f(ax)-f(bx)}{x}\dd{x}=\ln\left(\frac{a}{b}\right)\left[\lim_{t\to\infty} f(t)-f(0)\right].$$

\vspace{10pt}
\textbf{Nota: Aquí terminó la clase del 29 de julio del 2021.}
\newpage

\section*{Sucesiones de funciones}

\subsubsection*{Convergencia puntual}

Una sucesión de funciones $f_n:E\to\mathbb{R}$ con $E\subseteq\mathbb{R}$ converge puntualmente a $f(x)$ en $E_0\subseteq E$ si para todo $\varepsilon>0$ existe $N(\varepsilon,x)\in\mathbb{Z}^+$ tal que si $n\geq N$ entonces $|f_n(x)-f(x)|<\varepsilon$. Es decir
$$\lim_{n\to\infty}f_n(x)=f(x).$$
Nótese que si una sucesión de funciones continuas converge, no necesariamente converge a una función continua. Considerar, por ejemplo, $f_n:[0,1]\to\mathbb{R}$ tal que $f_n(x)=x^n$. Además, nótese también que si una sucesión de funciones diferenciables converge, tampoco es necesario que converja a una función diferenciable. En este caso, un contraejemplo está en $f_n:\mathbb{R}\to\mathbb{R}$ tal que $f(x)=\sqrt{x^2+1/n}$.

\subsubsection*{Convergencia uniforme}

Una sucesión de funciones $f_n:E\to\mathbb{R}$ con $E\subseteq\mathbb{R}$ converge uniformemente a $f(x)$ para todo $x$ en $E_0\subseteq E$ si para todo $\varepsilon>0$ existe $N(\varepsilon)\in\mathbb{Z}^+$ tal que si $n\geq N$ entonces $|f_n(x)-f(x)|<\varepsilon$.

\subsection*{Criterio de Cauchy para convergencia uniforme}

Sea $E\subseteq\mathbb{R}$ y sea $(f_n)$ una sucesión de funciones $f:E\to\mathbb{R}$. Entonces, la sucesión $(f_n)$ converge uniformemente a alguna función $f(x)$ en $E_0\subseteq E$ si y solo si para todo $\varepsilon>0$ existe $N(\varepsilon)\in\mathbb{Z}^+$ tal que si $m,n\geq N$ entonces $|f_m(x)-f_n(x)|<\varepsilon$, para todo $x\in E_0$.

\vspace{10pt}
\textbf{Nota: Aquí terminó la clase del 2 de agosto del 2021.}

\section*{Series numéricas}

\subsection*{Sucesión de sumas parciales}

Se dice que la serie numérica $\displaystyle \sum_{k=1}^{\infty}a_k$ forma la sucesión de sumas parciales dada por
$$S_n=\sum_{k=1}^{n}a_k=a_1+a_2+\cdots+a_n.$$

\subsection*{Convergencia de series}

Se dice que la serie numérica $\displaystyle \sum_{k=1}^{\infty}a_k$ converge si y solo si la sucesión de sumas parciales $(S_n)$ converge. Es decir, si y solo si 
$$\lim_{n\to\infty} S_n=L\in\mathbb{R}.$$

\subsubsection*{Criterio de divergencia}

Sea $\displaystyle \sum_{k=1}^{\infty}a_k$ convergente. Entonces, $\displaystyle\lim_{n\to\infty}a_n=0$. Por contrapuesta, esto equivale a que si el límite de una sucesión no es 0, su sumatoria es divergente.

\subsubsection*{Convergencia absoluta}

Se dice que la serie numérica $\displaystyle \sum_{k=1}^{\infty}a_k$ converge absolutamente si $\displaystyle \sum_{k=1}^{\infty}|a_k|$ converge.

\subsubsection*{Criterio de convergencia absoluta}

Si la serie numérica $\displaystyle \sum_{k=1}^{\infty}a_k$ converge absolutamente, entonces la serie como tal converge.

\vspace{10pt}
\textbf{Nota: Aquí terminó la clase del 5 de agosto del 2021.}

\subsubsection*{Criterio de comparación}

\begin{itemize}
\item Si la serie numérica $\displaystyle \sum_{k=1}^{\infty}b_k$ converge y $0\leq a_k\leq b_k$, entonces $\displaystyle \sum_{k=1}^{\infty}a_k$ converge.

\item Si la serie numérica $\displaystyle \sum_{k=1}^{\infty}a_k$ diverge y $0\leq a_k\leq b_k$, entonces $\displaystyle \sum_{k=1}^{\infty}b_k$ diverge.
\end{itemize}

\subsubsection*{Criterio de $p$-series}

La serie numérica $\displaystyle \sum_{k=1}^{\infty}\frac{1}{k^p}$ converge para $p>1$ y diverge para $p\leq 1$.

\subsubsection*{Criterio de la razón}

\begin{itemize}

\item Si $\displaystyle\lim_{n\to\infty}\left|\frac{a_{n+1}}{a_n}\right|$ existe y es menor que 1, la serie $\displaystyle \sum_{k=1}^{\infty}a_k$ converge absolutamente.

\item Si $\displaystyle\lim_{n\to\infty}\left|\frac{a_{n+1}}{a_n}\right|$ existe y es mayor que 1, o bien es infinito, la serie $\displaystyle \sum_{k=1}^{\infty}a_k$ diverge.

\item Si $\displaystyle\lim_{n\to\infty}\left|\frac{a_{n+1}}{a_n}\right|$ existe y es igual a 1, el criterio no es concluyente.

\end{itemize}

\vspace{10pt}
\textbf{Nota: Aquí terminó la clase del 9 de agosto del 2021.}

\subsubsection*{Criterio de condensación}

Sea $\displaystyle \sum_{k=1}^{\infty}a_k$ tal que $a_k>0$ y $(a_k)$ es decreciente. Esta converge si y solo si $\displaystyle \sum_{k=1}^{\infty}2^k\cdot a_{2^k}$ converge.

\subsubsection*{Criterio de la integral}

Suponga que $f(x)$ es una función integrable, positiva y decreciente en el intervalo cerrado $[1,\infty)$ y que $f(k)=a_k$ para todo $n$ entero positivo. Entonces,
$$\displaystyle \sum_{k=1}^{\infty}a_k \text{ converge } \iff \displaystyle\int_1^{\infty}f(x)\dd{x} \text{ converge.}$$

\newpage
\subsubsection*{Segundo criterio de la razón}

Considere la serie $\displaystyle \sum_{k=1}^{\infty}a_k$, y sean los límites
\begin{align*}
L_1=\lim_{n\to\infty}\left|\frac{a_{2n}}{a_n}\right|, && L_2=\lim_{n\to\infty}\left|\frac{a_{2n+1}}{a_n}\right|.
\end{align*}
\begin{itemize}
\item Si $L_1<1/2$ y $L_2<1/2$, entonces $\displaystyle \sum_{k=1}^{\infty}a_k$ converge.

\item Si $L_1>1/2$ y $L_2>1/2$, entonces $\displaystyle \sum_{k=1}^{\infty}a_k$ diverge.

\item En cualquier otro caso, el criterio no es concluyente.

\end{itemize}

\subsubsection*{Criterio de Raabe}

Sea $\displaystyle \sum_{k=1}^{\infty}a_k$ tal que $a_k>0$ y sea $L=\displaystyle\lim_{n\to\infty}n\left(\frac{a_n}{a_{n+1}}-1\right)$.
\begin{itemize}

\item Si $L>1$, la serie $\displaystyle \sum_{k=1}^{\infty}a_k$ converge.

\item Si $L<1$, la serie $\displaystyle \sum_{k=1}^{\infty}a_k$ diverge.

\item Si $L=1$, el criterio no es concluyente.

\end{itemize}

\vspace{10pt}
\textbf{Nota: Aquí terminó la clase del 12 de agosto del 2021.}

\subsubsection*{Criterio de Kummer}

Sea $\displaystyle \sum_{k=1}^{\infty}a_k$ y sea $\displaystyle \sum_{k=1}^{\infty}b_k$ una serie divergente tales que $a_k$ y $b_k$ son siempre positivos. Además, considérese
$$\alpha=\lim_{n\to\infty} \left(\frac{1}{b_n}\cdot\frac{a_n}{a_{n+1}}-\frac{1}{b_{n+1}}\right).$$
\begin{itemize}

\item Si $\alpha>0$, la serie $\displaystyle \sum_{k=1}^{\infty}a_k$ converge.

\item Si $\alpha<0$, la serie $\displaystyle \sum_{k=1}^{\infty}a_k$ diverge.

\item Si $\alpha=0$, el criterio no es concluyente.

\end{itemize}
Con $b_n=1$ para todo $n$, Kummer produce un criterio de la razón para series con términos positivos. Con $b_n=1/n$ para todo $n$, Kummer produce el criterio de Raave.

\newpage
\subsubsection*{Criterio de comparación en el límite}

Sean $\displaystyle \sum_{k=1}^{\infty}a_k$ y $\displaystyle \sum_{k=1}^{\infty}b_k$ series de términos positivos. Si se tiene 
$$\lim_{n\to\infty} \frac{a_n}{b_n}\in\mathbb{R}^{+},$$
entonces la convergencia o divergencia de ambas series es la misma.

\subsubsection*{Criterio de Leibniz para series alternantes}

Suponga que $(a_k)$ es una sucesión decreciente de números positivos cuyo límite en el infinito es 0. Entonces, $\displaystyle \sum_{k=1}^{\infty}(-1)^{k+1} a_k$ converge.

\vspace{10pt}
\textbf{Nota: Aquí terminó la clase del 16 de agosto del 2021.}

\subsubsection*{Criterio de la raíz}

Sea $\displaystyle \sum_{k=1}^{\infty}a_k$ y sea $L=\displaystyle\lim_{n\to\infty}\big(|a_n|\big)^{1/n}$. Entonces,
\begin{itemize}

\item Si $L<1$, la serie $\displaystyle \sum_{k=1}^{\infty}a_k$ converge absolutamente.

\item Si $L>1$, la serie $\displaystyle \sum_{k=1}^{\infty}a_k$ diverge.

\item Si $L=1$, el criterio no es concluyente.

\end{itemize}

\subsection*{Representación en series de funciones}

\subsubsection*{Convergencia de sucesiones de funciones}

Suponga que $\{f_n\}$ es una sucesión de funciones definida sobre el conjunto $E$, y suponga que la sucesión numérica $\{f_n(x)\}$ converge para cada $x\in E$. Entonces, se dice que $\{f_n\}$ converge a $f$ sobre $E$.

\subsubsection*{Convergencia de series de funciones}

Si $\displaystyle\sum_{n=1}^{\infty} f_n(x)$ converge para cada $x\in E$, se define a $f(x)$ como el límite de la suma.

\subsubsection*{$M$-test de Weierstrass}

Suponga que $\{f_n\}$ es una sucesión de funciones $f_n: E\to\mathbb{R}^+$, y suponga que $|f_n(x)|\leq M_n$ para $M_n\in\mathbb{R}$ y para todo $x\in E$. Si $\displaystyle\sum_{n=1}^\infty M_n$ converge, entonces $\displaystyle\sum_{n=1}^{\infty}f_n(x)$ converge uniformemente. 

\subsubsection*{Preservación de continuidad}

Si $\{f_n\}$ es una sucesión de funciones continuas $f_n: E\to\mathbb{R}$ que converge uniformemente a $f: E\to\mathbb{R}$, entonces la función $f$ también es continua.

\vspace{10pt}
\textbf{Nota: Aquí terminó la clase del 23 de agosto del 2021.}

\section*{Integral de Riemann-Stieltjes}

Sea $\alpha:[a,b]\to\mathbb{R}$ una función monótona creciente tal que $\alpha(a)$ y $\alpha(b)$ sean finitos. Entonces, $\alpha(x)$ es acotada. Considérese $P\in P[a,b]$ y hagamos $\Delta \alpha_i=\alpha(x_i)-\alpha(x_{i-1})$, para $a\leq x_{i-1}\leq x_i\leq b$. Para cualquier función acotada $f:[a,b]\to\mathbb{R}$, se define:
\begin{align*}
M_k(f)=\sup\{f(x): x\in[x_{k-1},x_k]\}, && m_k(f)=\inf\{f(x): x\in[x_{k-1},x_k]\},
\end{align*}
con lo cual...
\begin{align*}
U(P,f,\alpha)=\sum_{k=1}^{n}M_k(f)\Delta \alpha_{k}, && L(P,f,\alpha)=\sum_{k=1}^{n}m_k(f)\Delta \alpha_{k}.
\end{align*}
Además, se define también:
\begin{align*}
\overline{\int_{a}^{b}}f\dd{\alpha}=\inf\big\{U(P,f,\alpha): P\in P[a,b]\big\}, && \underline{\int_{a}^{b}}f\dd{\alpha}=\sup\big\{L(P,f,\alpha): P\in P[a,b]\big\}.
\end{align*}
Se dice que $f$ es integrable con respecto a $\alpha$ en el sentido de Riemann-Stieltjes, lo cual se denota $f\in R(\alpha)$, si y solo si se cumple
$$\overline{\int_{a}^{b}}f\dd{\alpha}=\underline{\int_{a}^{b}}f\dd{\alpha}:=\int_{a}^{b}f(x)\dd{\alpha(x)}.$$
Esta es la integral de Riemann-Stieltjes. Nótese que si $\alpha(x)=x$, esta equivale a la integral de Riemann.

\subsection*{Teorema de preservación de integrabilidad R-S}

Sea $\alpha:[a,b]\to\mathbb{R}$ una función monótona creciente. Suponga que $(f_n)$ es una sucesión de funciones tal que $f_n\in R(\alpha)$ para todo entero positivo $n$, y que la sucesión converge uniformemente a $f$ sobre $[a,b]$. Entonces, $f\in R(\alpha)$ sobre $[a,b]$ y:
$$\lim_{n\to\infty}\int_a^b f_n\dd{\alpha}=\int_a^b f\dd{\alpha}.$$

\subsubsection*{Corolario del teorema de preservación de integrabilidad R-S}

Suponga que $(f_n)$ es una sucesión de funciones tal que $f_n\in R(\alpha)$ para todo entero positivo $n$, y que
$$f(x)=\sum_{n=1}^{\infty}f_n(x)$$
uniformemente sobre $[a,b]$, entonces
$$\sum_{n=1}^{\infty}\int_a^b f_n(x)=\int_a^b f(x)\dd{x}.$$

\vspace{10pt}
\textbf{Nota: Aquí terminó la clase del 30 de agosto del 2021.}

\newpage
\section*{Norma del supremo y convergencia}

\subsection*{Norma del supremo}

Sea $X$ un espacio métrico y sea $C(X)$ el conjunto de todas las funciones continuas y acotadas sobre $X$. Para $f\in C(X)$, se define la norma del supremo de $f$ sobre $X$ como:
$$\norm{f}=\sup_{x\in X}|f(x)|.$$
Nótese que...
\begin{itemize}
\item Si $f$ es acotada, entonces $|f(x)|$ es finito para todo $x\in X$. Además, $\norm{f}=0$ si y solo si $f$ es la función 0.

\item Por desigualdad triangular, $|f+g|\leq|f|+|g|$. Entonces, $\norm{f+g}\leq\norm{f}+\norm{g}$.

\item Si $f,g\in C(X)$, se define $d(f,g):=\norm{f-g}$.

\item En algunos casos, a los cerrados de $C(X)$ se les llama uniformemente cerrados. A la cerradura de un subconjunto de $C(X)$ se le llama cerradura uniforme.

\item Teorema: $\big(C(X),d\big)$ es un espacio métrico completo (toda sucesión de Cauchy en él es convergente).
\end{itemize}

\subsection*{Norma en espacios vectoriales}

Sea $(X,+,\cdot,\mathbb{R})$ un espacio vectorial. Una función $\norm{\cdot}:X\to[0,\infty)$ es una norma sobre $X$ si...
\begin{itemize}
\item Para todo $x,y\in X$, se cumple $\norm{x+y}\leq\norm{x}+\norm{y},$;

\item Para todo $x\in X$ y $\lambda\in\mathbb{R}$ se cumple $\norm{\lambda x}=|\lambda|\norm{x}$;

\item Si $x\in X$, entonces $\norm{x}=0$ si y solo si $x=0$;
\end{itemize}
Se llama espacio normado al par ordenado $(X,\norm{\cdot})$.

\subsection*{Convergencia de sucesiones}

Una sucesión $(x_n)\subset X$ es convergente a $x\in X$ si $\norm{x_n-x}\to 0$.

\subsection*{Sucesiones de Cauchy}

Una sucesión $(x_n)\subset X$ es de Cauchy si para todo $\varepsilon>0$ existe $N\in\mathbb{Z}^+$ tal que si $m$ y $n$ son enteros mayores que $N$, entonces $|x_m-x_n|<\varepsilon$.

\begin{itemize}
\item Se dice que $X$ es completo si toda sucesión de Cauchy es convergente.

\item Un espacio vectorial normado y completo es un espacio de Banach.

\item Un espacio vectorial de Banach cuya norma proviene de un producto interno (donde $\norm{x}=\sqrt{\langle x,x\rangle}$) es un espacio de Hilbert.
\end{itemize}

\subsection*{Reales completados}

El conjunto de los reales completados es el conjunto $\overline{\mathbb{R}}=\mathbb{R}\cup\{-\infty,\infty\}$.

\subsection*{Norma del supremo en $\mathbb{R}$}

Sea $f$ una función definida en un subconjunto $A$ de $\mathbb{R}$. La norma del supremo o norma uniforme de $f$ es el número sobre $\overline{\mathbb{R}}$ definido por $\norm{f}_{\infty}=\sup\{|f(x)|: x\in A\}.$

\subsection*{Convergencia de sucesiones en $\overline{\mathbb{R}}$}

Una sucesión $(x_n)$ de elementos en $\overline{\mathbb{R}}$ converge a $x\in\mathbb{R}$ si existe $N\in\mathbb{Z}^+$ tal que $x_n$ es finito para todo $n\geq N$, y si la sucesión correspondiente de $(x_n)$ de elementos en $\mathbb{R}$ converge a $x$.

\subsection*{Caracterización de convergencia uniforme}

Sea $(f_n)$ una sucesión de funciones definidas en un subconjunto de los reales y sea $f$ una función definida en el mismo subconjunto. Entonces, $(f_n)$ converge uniformemente a $f$ si y solo si la sucesión $(\norm{f_n-f}_{\infty})$ sobre $\overline{\mathbb{R}}$ converge a 0.

\subsection*{Teorema: Critero de Cauchy uniforme}

Sea $(f_n)$ una sucesión de funciones definidas en un subconjunto $A$ de los reales. Entonces, $(f_n)$ converge uniformemente a una función $f$ si y solo si, para todo $\epsilon>0$ existe $N\in\mathbb{Z}^+$ tal que si $m$ y $n$ son enteros mayores que $N$, entonces $|f_m(x)-f_n(x)|<\varepsilon$ para todo $x\in A$, si y solo si, para todo $\epsilon>0$ existe $N\in\mathbb{Z}^+$ tal que si $m$ y $n$ son enteros mayores que $N$, entonces $\norm{f_m-f_n}<\varepsilon$.

\subsection*{Teorema de preservación de continuidad puntual}

Sea $(f_n)$ una sucesión de funciones definidas en un subconjunto de los reales que converge a $f$, una función definida en el mismo subconjunto. Si cada $f_n$ es continua en un punto $x\in A$, entonces $f$ es continua en $x\in A$.

\vspace{10pt}
\textbf{Nota: Aquí terminó la clase del 2 de septiembre del 2021.}

\subsection*{Otro teorema de Weierstrass}

Sea $(f_n)$ una sucesión de funciones diferenciables $f_n:(a,b)\to\mathbb{R}$ que converge puntualmente a una función $f$ y donde $(f_n')$ converge uniformemente a una función $g$. Entonces, $f$ es diferenciable sobre $(a,b)$ y $f'=g$.

\subsection*{Monotonía de sucesiones de funciones}

\begin{itemize}
\item Se dice que $(f_n)$ es creciente si $f_n(x)\leq f_{n+1}(x)$ para todo $n\in\mathbb{Z}^{+}$ y para todo $x$ en el dominio de $f$.

\item Se dice que $(f_n)$ es decreciente si $f_n(x)\geq f_{n+1}(x)$ para todo $n\in\mathbb{Z}^{+}$ y para todo $x$ en el dominio de $f$.

\item Si $(f_n)$ es creciente o decreciente, entonces la sucesión es monótona.
\end{itemize}

\subsection*{Teorema de Dini (cuasi-converso al teorema anterior)}

Sea $(f_n)$ una sucesión de funciones definidas en un intervalo cerrado y acotado $I$ de los reales. Suponga que $(f_n)$ converge puntualmente a una función continua $f$, y que $(f_n)$ es una sucesión monótona. Entonces, $(f_n)$ converge uniformemente a $f$.

\vspace{10pt}
\textbf{Nota: Aquí terminó la clase del 6 de septiembre del 2021.}

\section*{¿Topología?}

\subsection*{Vecindad}

Sea $(X,\tau)$ un espacio topológico y sea $p\in X$. Una vecindad de $p$ es cualquier conjunto $U$ tal que $p\in U$ y para el cual existe $V\in\tau$ tal que $V\subseteq U$ y $p\in V$.

\subsection*{Sucesión en espacios topológicos}

Sea $(X,\tau)$ un espacio topológico. Una sucesión sobre $X$ es cualquier función $f:\mathbb{Z}^+\to X$.

\subsection*{Convergencia en espacios topológicos}

Se dice que la sucesión $(x_n)$ en el espacio topológico $(X,\tau)$ converge a $p\in X$ si cada vecindad de $x$ contiene a la cola de la sucesión. Es decir, para toda vecindad $U$ de $p\in X$ existe $N\in\mathbb{Z}^+$ tal que para todo $n\geq N$ se cumple $x_n\in U$.

\subsection*{Conjuntos dirigidos}

Un conjunto $D$ es dirigido si existe una relación binaria $\leq$ sobre $D$ tal que:
\begin{itemize}
\item Es una relación reflexiva;

\item Es una relación transitiva;

\item Si $m,n\in D$, entonces existe $r\in D$ tal que $m\leq r$ y $n\leq r$.
\end{itemize}
A la relación $\leq$ se le llama dirección. La colección de todas las vecindades de un $x\in X$ para un espacio topológico $(X,\tau)$ es un conjunto dirigido con dirección $\subseteq$.

\subsection*{Redes}

Una red en un conjunto $X$ es una función $f:D\to X$, donde $D$ es un conjunto dirigido.

\vspace{10pt}
\textbf{Nota: Aquí terminó la clase del 9 de septiembre del 2021.}
\newpage

\section*{Diferenciación en $\mathbb{R}^n$}

\subsection*{Diferenciación de $\mathbb{R}$ a $\mathbb{R}$}

Sea $X\subseteq\mathbb{R}$. Una función $F:X\to\mathbb{R}$ es diferenciable en $a\in X$ si la gráfica de $F$ tiene una recta tangente en el punto $(a,F(a))$. La ecuación de esta recta tangente es $H(x)=F'(a)[x-a]+F(a)$. Es decir, ser diferenciable puede ser visto como que $H(x)$ existe, o que su pendiente $F'(a)$ exista. Nótese que el comportamiento de $H$ y de $F$ en $a$ es exactamente el mismo (sus valores y sus pendientes son iguales). Esto es, $H$ aproxima a $F$ en una vecindad de $a$.

\subsection*{Superficies}

Sea $X\subseteq\mathbb{R}^2$ un abierto. La gráfica de una función $f:X\to\mathbb{R}$ es llamada una superficie. El plano tangente en un punto $(a,b)\in X$ de la superficie esta dado por
$$H(x,y)=\frac{\partial f}{\partial x}(a,b)[x-a]+\frac{\partial f}{\partial y}(a,b)[y-b]+f(a,b).$$
Esto es, $H(x,y)=\nabla f(a,b)\cdot\displaystyle\begin{bmatrix} x-a \\ y-b \end{bmatrix}+f(a,b)$. De nuevo, nótese que
\begin{align*}
H(a,b)=f(a,b) && \frac{\partial h}{\partial x}(a,b)=\frac{\partial f}{\partial x}(a,b) && \frac{\partial h}{\partial y}(a,b)=\frac{\partial f}{\partial y}(a,b).
\end{align*}
Sin embargo, la existencia de las derivadas parciales en estas funciones no equivale a la existencia del plano tangente. Las derivadas parciales pueden existir aunque no haya un plano tangente. Un ejemplo es la función $f(x,y)=\big||x|-|y|\big|-|x|-|y|$ en el punto $(0,0)$.

\subsection*{Diferenciabilidad de $\mathbb{R}^2$ a $\mathbb{R}$}

Sea $X\subseteq\mathbb{R}^2$ un abierto, y sea $f:X\to\mathbb{R}$. Se dice que $f$ es diferenciable en $(a,b)\in X$ si existen las derivadas parciales en el punto ($f_x(a,b)$ y $f_y(a,b)$) y si la función $H(x,y)$ descrita anteriormente es una buena aproximación lineal de $f$. Es decir,
$$\lim_{(x,y)\to(a,b)}\frac{f(x,y)-H(x,y)}{\norm{(x,y)-(a,b)}}=0.$$

\vspace{10pt}
\textbf{Nota: Aquí terminó la clase del 20 de septiembre del 2021.}

\subsection*{Diferenciabilidad de $\mathbb{R}^n$ a $\mathbb{R}$}

Sea $X\subseteq\mathbb{R}^n$ un abierto, y sea $f:X\to\mathbb{R}$. Se dice que $f$ es diferenciable en $\vec{a}=(a_1,a_2,\ldots,a_n)\in X$ si existen todas las derivadas parciales en el punto ($f_k(\vec{a})$ para $1\leq k\leq n$ entero) y si la función
$$H(\vec{x})=\nabla f(\vec{a})\cdot\begin{bmatrix} x_1-a_1 \\ x_2-a_2 \\ \vdots \\ x_n-a_n \end{bmatrix}+f(\vec{a})=\frac{\partial f}{\partial x_1}(\vec{a})[x_1-a_1]+\frac{\partial f}{\partial x_2}(\vec{a})[x_2-a_2]+\cdots+\frac{\partial f}{\partial x_n}(\vec{a})[x_n-a_n]+f(\vec{a})$$
es una buena aproximación lineal de $f$. Es decir,
$$\lim_{\vec{x}\to\vec{a}}\frac{f(\vec{x})-H(\vec{x})}{\norm{\vec{x}-\vec{a}}}=0.$$

\newpage
\subsubsection*{Derivada de $\mathbb{R}^n$ a $\mathbb{R}$}

En funciones de $\mathbb{R}^n$ a $\mathbb{R}$ se entiende por derivada en un punto $\vec{a}$ a la matriz fila $Df(\vec{a})$ cuyas entradas son las componentes del gradiente evaluado en el punto, $\nabla f(\vec{a})$. No hay que confundir al gradiente con esta matriz: una matriz fila no es el mismo objeto que un vector. La gráfica de estas funciones es una hipersuperficie en $\mathbb{R}^{n+1}$. Si $f$ es diferenciable en $\vec{a}$, entonces la hipersuperficie determinada por la gráfica de $f$ tiene un hiperplano tangente en $\big(\vec{a},f(\vec{a})\big)$. Este hiperplano está dado por $H(\vec{x})$.
 
\subsection*{Diferenciabilidad de $\mathbb{R}^n$ a $\mathbb{R}^m$}

Sea $X\subseteq\mathbb{R}^n$ un abierto, y sea $f:X\to\mathbb{R}^m$ una función vectorial, cuya imagen se presenta como
$$f(\vec{x})=\big(f_1(\vec{x}),f_2(\vec{x}),\ldots,f_m(\vec{x})\big).$$
De esto, se define la matriz (Jacobiano) de $m\times n$ de derivadas parciales a continuación:
$$Df(\vec{x})=
\begin{pmatrix}
\displaystyle\frac{\partial f_1}{\partial x_1}(\vec{a}) & \displaystyle\frac{\partial f_1}{\partial x_2}(\vec{a}) & \cdots & \displaystyle\frac{\partial f_1}{\partial x_n}(\vec{a})\\
\displaystyle\frac{\partial f_2}{\partial x_1}(\vec{a}) & \displaystyle\frac{\partial f_2}{\partial x_2}(\vec{a}) & \cdots & \displaystyle\frac{\partial f_2}{\partial x_n}(\vec{a})\\
\vdots & \vdots & \ddots & \vdots\\
\displaystyle\frac{\partial f_m}{\partial x_1}(\vec{a}) & \displaystyle\frac{\partial f_m}{\partial x_2}(\vec{a}) & \cdots & \displaystyle\frac{\partial f_m}{\partial x_n}(\vec{a})
\end{pmatrix}.$$
Se dice que $f$ es diferenciable en $\vec{a}$ si $Df(\vec{a})$ existe y si la función $H:\mathbb{R}^n\to\mathbb{R}^m$ definida por $H(\vec{x})=f(\vec{a})+Df(\vec{a})[\vec{x}-\vec{a}]$ es una buena aproximación lineal de $f$ cerca de $\vec{a}$. Es decir, se cumple
$$\lim_{\vec{x}\to\vec{a}}\frac{f(\vec{x})-H(\vec{x})}{\norm{\vec{x}-\vec{a}}}=0.$$

\subsection*{T(A): Diferenciabilidad implica continuidad}

Sea $X\subseteq\mathbb{R}^n$ un abierto, y sea $f:X\to\mathbb{R}^m$ una función vectorial diferenciable en $\vec{a}\in X$. Entonces, $f$ es continua en $\vec{a}$.

\subsection*{T(B): Diferenciabilidad de las funciones componente en cada variable implica diferenciabilidad}

Sea $X\subseteq\mathbb{R}^n$ un abierto, y sea $f:X\to\mathbb{R}^m$ una función vectorial tal que todas las derivadas parciales de las funciones componentes de las imágenes de $f$ existen y son continuas en un punto $\vec{a}\in X$, entonces $f$ es diferenciable en $\vec{a}$.

\subsection*{T(C): Caracterización de funciones diferenciables}

Sea $X\subseteq\mathbb{R}^n$ un abierto. Una función $f:X\to\mathbb{R}^m$ es diferenciable en $\vec{a}\in X$ si y solo si cada función componente $f_k:X\to \mathbb{R}^m$ para $k\in\{1,2,\ldots,m\}$ es diferenciable en $\vec{a}$.

\vspace{10pt}
\textbf{Nota: Aquí terminó la clase del 23 de septiembre del 2021.}

\subsection*{Lema para T(A): Acotación de norma}

Sean $\vec{x}\in\mathbb{R}^n$ y $B\in R^{m\times n}$ con entradas $b_{i,j}$. Si $y=B\vec{x}$, entonces $\norm{y}\leq K\norm{x}$, con $K=\displaystyle\sqrt{\sum_{i\leq m,j\leq n}b_{i,j}^2}$.

\vspace{10pt}
\textbf{Nota: Aquí terminó la clase del 27 de septiembre del 2021.}

\subsection*{Cerradura de la suma en las funciones diferenciables}

Sea $X\subseteq\mathbb{R}^n$ y sean $f,g: X\to\mathbb{R}^m$ funciones diferenciables en $\vec{a}\in X$, y sea $\alpha\in\mathbb{R}$. Entonces, la función $f+g$ es diferenciable en $\vec{a}$, y se tiene que $D(f+g)(\vec{a})=Df(\vec{a})+Dg(\vec{a})$. Además, $J(\vec{x})=\alpha f(\vec{x})$ es diferenciable en $\vec{a}$, y se tiene que $DJ(\vec{x})=\alpha Df(\vec{x})$.

\subsection*{Reglas del producto y cociente en las funciones diferenciables de $\mathbb{R}^n$ a $\mathbb{R}$}

Sea $X\subseteq\mathbb{R}^n$ y sean $f,g: X\to\mathbb{R}$ funciones diferenciables en $\vec{a}\in X$. Entonces, la función $fg$ es diferenciable en $\vec{a}$, y se tiene que $D(fg)(\vec{a})=Df(\vec{a})g(\vec{a})+f(\vec{a})Dg(\vec{a})$. Además, si $g(\vec{a})\neq0$, la función $f/g$ es diferenciable en $\vec{a}$, y se tiene que
$$D\left(\frac{f}{g}\right)(\vec{a})=\frac{g(\vec{a})Df(\vec{a})-f(\vec{a})Dg(\vec{a})}{[g(\vec{a})]^2}.$$
Si $\overline{g}:X\to\mathbb{R}^m$ es también diferenciable en $\vec{a}\in X$, entonces $f\overline{g}$ es diferenciable en $\vec{a}$, y se cumple que 
$$D(f\overline{g})(\vec{a})=\overline{g}(\vec{a})Df(\vec{a})+f(\vec{a})D\overline{g}(\vec{a}).$$ 


\vspace{10pt}
\textbf{Nota: Aquí terminó la clase del 4 de octubre del 2021.}

\subsection*{Derivada direccional}

Sea $X$ un abierto en $\mathbb{R}^n$, sea $f:X\to\mathbb{R}$ una función escalar, y sea $\vec{a}\in X$. Si $\hat{v}\in\mathbb{R}^n$ es un vector unitario, entonces la derivada direccional de $f$ en $\vec{a}$, en la dirección de $\hat{v}$, está dada por
$$D_{\hat{v}}f(\vec{a})=\lim_{h\to0}\frac{f(\vec{a}+h\hat{v})-f(\vec{a})}{h}.$$
Si $f$ es diferenciable, entonces sus derivadas parciales determinan las derivadas direccionales en cualquier dirección $\hat{v}$. De esta manera, si $F(t)=f(\vec{a}+t\hat{v})$, entonces $D_{\hat{v}}f(\vec{a})=F'(0)$. Esto es,
$$D_{\hat{v}}f(\vec{a})=\left[\dv{t}f(\vec{a}+t\hat{v})\right]_{t=0.}$$

\subsection*{Regla de la cadena en $\mathbb{R}^2$}

Sean $T\subseteq\mathbb{R}$ y $X\subseteq\mathbb{R}^2$ abiertos, sea $x:T\to\mathbb{R}^2$ diferenciable en $t_0\in T$, y sea $f:X\to\mathbb{R}$ diferenciable en $\vec{x_0}=x(t_0)=(x_0,y_0)\in X$. También supóngase que la imagen de $X$ bajo $f$ es un subconjunto de $X$, y que además $f$ tiene primera derivada continua. Entonces, $f\circ x: T\to\mathbb{R}$ es diferenciable en $t_0$, y se cumple que
$$\frac{\dd{f}}{\dd{t}}(t_0)=\frac{\partial{f}}{\partial{x}}(\vec{x_0})\cdot\frac{\dd{x}}{\dd{t}}(t_0)+\frac{\partial{f}}{\partial{y}}(\vec{x_0})\cdot\frac{\dd{y}}{\dd{t}}(t_0).$$

\subsection*{Regla de la cadena en $\mathbb{R}^n$}

En general, si $T\subseteq\mathbb{R}$ y $X\subseteq\mathbb{R}^n$ son abiertos, y se tienen $f:X\to\mathbb{R}$ y $x:T\to\mathbb{R}^n$, entonces
$$\frac{\dd{f}}{\dd{t}}(t_0)=
\begin{bmatrix}
\displaystyle\frac{\partial{f}}{\partial{x_1}}(\vec{x_0})& \displaystyle\frac{\partial{f}}{\partial{x_2}}(\vec{x_0})& \cdots &\displaystyle\frac{\partial{f}}{\partial{x_n}}(\vec{x_0})
\end{bmatrix}\cdot
\begin{bmatrix}
\displaystyle\frac{\dd{x_1}}{\dd{t}}(t_0)\\ \vdots \\ \displaystyle\frac{\dd{x_n}}{\dd{t}}(t_0)
\end{bmatrix}=Df(\vec{x_0})\cdot Dx(t_0)=\nabla f(\vec{x_0})\cdot \vec{x}'(t_0).$$

\subsection*{Existencia y simplificación de la derivada direccional}

Sea $X\subseteq\mathbb{R}^n$, y suponga que $f:X\to\mathbb{R}$ es diferenciable en $\vec{a}\in X$. Entonces, la derivada direccional de $f$ existe para cada dirección $\hat{v}\in\mathbb{R}^n$, y se tiene que $D_{\hat{v}}f(\vec{a})=\nabla f(\vec{a})\cdot\hat{v}$. Sin embargo, la existencia de las derivadas direccionales no aseguran la diferenciabilidad de una función.

\vspace{10pt}
\textbf{Nota: Aquí terminó la clase del 11 de octubre del 2021.}

\subsection*{Optimización de la derivada direccional}

Dado que $D_{\hat{v}}f(\vec{a})=\nabla f(\vec{a})\cdot\hat{v}$, entonces $D_{\hat{v}}f(\vec{a})$ se optimiza cuando $\hat{v}$ está en la dirección de $\nabla f$ y se minimiza cuando $\hat{v}$ está en la dirección opuesta a $\nabla f$. En efecto, $\norm{D_{\hat{v}}f(\vec{a})}=\norm{\nabla f(\vec{a})}$.

\subsection*{Superficie (visión informal)}

Una superficie se puede pensar como una función diferenciable $f:X\to\mathbb{R}$, donde $X\subseteq\mathbb{R}^2$ es un conjunto conexo y abierto. De esta manera, se grafican sobre el eje $z$ las imágenes de $X$ bajo $f$.

\subsection*{Superficie (visión formal)}

Otra manera de pensar en las superficies como funciones diferenciables $f:X\to\mathbb{R}^3$, donde $X\subseteq\mathbb{R}^2$ es un conjunto conexo y abierto. Esta visión corresponde a la parametrización de una función, para la cual $x$ y $y$ son funciones dependientes de dos parámetros $s$ y $t$. Es decir, cada punto de la superficie estaría dado por $f(x(s,t),y(s,t))=(f_1(x,y),f_2(x,y),f_3(x,y))$.

\subsection*{Plano tangente}

Considérese a una superficie $F(x,y,z)=0$ tal que $\nabla F\neq 0$	. El plano que pasa por un punto $P=(x_0,y_0,z_0)$ y es normal a $\nabla F(x_0,y_0,z_0)$ se le llama plano tangente en el punto $P$. La ecuación del plano tangente es $\nabla F(x_0,y_0,z_0)\cdot (x-x_0,y-y_0,z-z_0)=0$.

\vspace{10pt}
\textbf{Nota: Aquí terminó la clase del 14 de octubre del 2021.}

\subsection*{Relación entre superficies tangentes y polinomios de Taylor}

\subsubsection*{Planos tangentes en $\mathbb{R}^2$}

Sea $X\subseteq\mathbb{R}^2$ un abierto y sea $f:X\to\mathbb{R}$ una función de dos variables cuya primera derivada es continua. Entonces, cerca de $(a,b)\in X$, la mejor aproximación lineal de $f$ está dada por el planto tangente al punto $(a,b,f(a,b))$. Este plano está dado por:
$$P_1(x,y)=\frac{\partial f}{\partial x}(a,b)[x-a]+\frac{\partial f}{\partial y}(a,b)[y-b]+f(a,b).$$

\subsubsection*{Hiperplanos tangentes en $\mathbb{R}^n$}

Sea $X\subseteq\mathbb{R}^n$ un abierto y sea $f:X\to\mathbb{R}$ una función de $n$ variables cuya primera derivada es continua. Entonces, cerca de $\vec{a}\in X$, la mejor aproximación lineal de $f$ está dada por el hiperplanto tangente al punto $(a_1,a_2,\ldots,a_n,f(\vec{a}))$. Este plano está dado por:
$$P_1(\vec{x})=f(\vec{a})+\sum_{k=1}^n \frac{\partial f}{\partial x_k}(\vec{a})[x_k-a_k]=f(\vec{a})+\nabla f(\vec{a})\cdot[\vec{x}-\vec{a}].$$

\subsubsection*{Teorema de Taylor lineal}

Sea $X\subseteq\mathbb{R}^n$ un abierto y sea $f:X\to\mathbb{R}$ una función diferenciable en $\vec{a}\in X$. Si se define a la función $P_1(\vec{x})=f(\vec{a})+\nabla f(\vec{a})\cdot[\vec{x}-\vec{a}]$, entonces $f(\vec{x})=P_1(\vec{x})+R_1(\vec{x},\vec{a})$, donde $R_1(\vec{x},\vec{a})$ cumple con
$$\lim_{\vec{x}\to\vec{a}}\frac{R_1(\vec{x},\vec{a})}{\norm{\vec{x}-\vec{a}}}=0.$$

\subsubsection*{Cambio incremental}

Sea $X\subseteq\mathbb{R}^n$ un abierto, sea $f:X\to\mathbb{R}$ una función, y sea $\vec{a}\in X$. Entonces, el cambio incremental de $f$, denotado $\Delta f$, está dado por $\Delta f=f(\vec{a}+\vec{h})-f(\vec{a})$.

\subsubsection*{Diferencial total}

El diferencial total del $f$, denotado $\dd{f}$, está dado por $\dd{f}=\nabla f(\vec{a})\cdot[\vec{x}-\vec{a}]$. Cuando $\vec{h}\to 0$, se cumple que $\Delta f\to \dd{f}$. 

\subsubsection*{Aproximaciones cuadráticas en $\mathbb{R}^2$}

Sea $X\subseteq\mathbb{R}^2$ un abierto y sea $f:X\to\mathbb{R}$ una función de dos variables cuya primera y segunda derivada es continua. Entonces, cerca de $(a,b)\in X$, la mejor aproximación cuadrática de $f$ está dada por el polinomio de Taylor con dos términos, que es:
\begin{align*}
P_2(x,y)&=f(a,b)+f_x(a,b)[x-a]+f_y(a,b)[y-b]+\\
&+\frac{1}{2}f_{xx}(a,b)[x-a]^2+f_{xy}(a,b)[x-a][y-b]+\frac{1}{2}f_{yy}(a,b)[y-b]^2.
\end{align*}

\subsubsection*{Teorema de Taylor cuadrático}

Sea $X\subseteq\mathbb{R}^n$ un abierto y sea $f:X\to\mathbb{R}$ una función diferenciable dos veces en $\vec{a}\in X$. Si se define a la función 
$$P_2(\vec{x})=f(\vec{a})+\sum_{k=1}^n f_{x_k}(\vec{a})[x_k-a_k]+\frac{1}{2}\sum_{1\leq i,j\leq n} f_{x_ix_j}(\vec{a})[x_i-a_i][x_j-a_j],$$
entonces $f(\vec{x})=P_2(\vec{x})+R_2(\vec{x},\vec{a})$, donde $R_2(\vec{x},\vec{a})$ cumple con
$$\lim_{\vec{x}\to\vec{a}}\frac{R_2(\vec{x},\vec{a})}{\norm{\vec{x}-\vec{a}}^2}=0.$$

\vspace{10pt}
\textbf{Nota: Aquí terminó la clase del 18 de octubre del 2021.}

\subsection*{La matriz Hessiana}

La matriz Hessiana de una función $f:X\to\mathbb{R}$, donde $X\subseteq\mathbb{R}^n$ es un abierto, es la matriz cuya $ij$-ésima entrada está dada por $\displaystyle\frac{\partial^2f}{\partial x_j\partial x_i}=f_{x_ix_j}$, con $1\leq i,j\leq n$. Así,
$$Hf=\begin{pmatrix}
f_{x_1x_1} & f_{x_1x_2} & \cdots & f_{x_1x_n}\\
f_{x_2x_1} & f_{x_2x_2} & \cdots & f_{x_2x_n}\\
\vdots & \vdots & \ddots & \vdots\\
f_{x_nx_1} & f_{x_nx_2} & \cdots & f_{x_nx_n}
\end{pmatrix}.$$

\subsubsection*{Polinomio cuadrático de Taylor con la matriz Hessiana}

Nótese que $P_2(\vec{x})=f(\vec{a})+\displaystyle\sum_{i=1}^nf_{x_i}(\vec{a})h_i+\frac{1}{2}\sum_{1\leq i,j\leq n} f_{x_ix_j}(\vec{a})h_ih_j$, donde $\vec{h}=\vec{x}-\vec{a}$. Esto es:
\begin{align*}
P_2(\vec{x})&=f(\vec{a})+
\begin{bmatrix}
f_{x_1}(\vec{a}) &  f_{x_2}(\vec{a}) & \cdots f_{x_n}(\vec{a})
\end{bmatrix}\cdot\begin{bmatrix}
h_1\\ h_2\\ \vdots \\ h_n
\end{bmatrix}\\
&+\frac{1}{2}\begin{bmatrix}
h_1& h_2& \cdots & h_n
\end{bmatrix}\cdot\begin{bmatrix}
f_{x_1x_1} & f_{x_1x_2} & \cdots & f_{x_1x_n}\\
f_{x_2x_1} & f_{x_2x_2} & \cdots & f_{x_2x_n}\\
\vdots & \vdots & \ddots & \vdots\\
f_{x_nx_1} & f_{x_nx_2} & \cdots & f_{x_nx_n}
\end{bmatrix}\cdot\begin{bmatrix}
h_1\\ h_2\\ \vdots \\ h_n
\end{bmatrix}.
\end{align*}
De esta manera, simplificando la notación, $P_2(\vec{x})=f(\vec{a})+Df(\vec{a})\vec{h}+\displaystyle\frac{1}{2}\vec{h}^T\cdot Hf(\vec{a})\cdot\vec{h}$.

\subsubsection*{Polinomio cúbico de Taylor}

Sea $X\subseteq\mathbb{R}^n$ un abierto y sea $f:X\to\mathbb{R}$ una función diferenciable tres veces en $\vec{a}\in X$. Si se define a la función 
\begin{align*}
P_3(\vec{x})&=f(\vec{a})+\sum_{k=1}^n f_{x_k}(\vec{a})[x_k-a_k]+\frac{1}{2}\sum_{1\leq i,j\leq n} f_{x_ix_j}(\vec{a})[x_i-a_i][x_j-a_j]\\
&+\frac{1}{6}\sum_{1\leq i,j,k\leq n}f_{x_ix_jx_k}(\vec{a})[x_i-a_i][x_j-a_j][x_k-a_k],
\end{align*}
entonces esta es la mejor aproximación cúbica de $f$ en una vecindad de $\vec{a}$. Además se tiene que $f(\vec{x})=P_3(\vec{x})+R_2(\vec{x},\vec{a})$, donde $R_3(\vec{x},\vec{a})$ es el residuo que cumple con
$$\lim_{\vec{x}\to\vec{a}}\frac{R_3(\vec{x},\vec{a})}{\norm{\vec{x}-\vec{a}}^3}=0.$$

\subsection*{Optimización en $\mathbb{R}^n$}

\subsubsection*{Máximos y mínimos locales}

Sea $X\subseteq\mathbb{R}^n$ un abierto y sea $f:X\to\mathbb{R}$ una función escalar. Se dice que $f$ tiene un mínimo local en $\vec{a}\in X$ si existe una vecindad $U$ de $\vec{a}$ tal que $f(\vec{x})\geq f(\vec{a})$ para todo $\vec{x}\in U$. Se dice que tiene un máximo local si $f(\vec{x})\leq f(\vec{a})$ para todo $\vec{x}\in U$.

\subsubsection*{Puntos críticos}

Un punto $\vec{a}$ del dominio de $f$, donde $Df(\vec{a})$ es cero o no existente, es un punto crítico de $f$.

\subsubsection*{Teorema de extremos locales y puntos críticos}

Sea $X\subseteq\mathbb{R}^n$ un abierto y sea $f:X\to\mathbb{R}$ una función escalar diferenciable. Si $f$ tiene un extremo local en $\vec{a}\in X$, entonces $Df(\vec{a})=0$. Esto es, los extremos locales ocurren en puntos críticos.

\vspace{10pt}
\textbf{Nota: Aquí terminó la clase del 21 de octubre del 2021.}

\subsection*{Formas bilineales}

Sea $V$ un espacio vectorial finito dimensional sobre $\mathbb{R}$. Una forma bilineal sobre $V$ es un mapeo $f:V\times V\to\mathbb{R}$ tal que $f(\alpha \vec{x}+\beta \vec{y},\vec{z})=\alpha f(\vec{x},\vec{z})+\beta f(\vec{y},\vec{z})$ y que $f(\vec{x},\alpha \vec{y}+\beta \vec{z})=\alpha f(\vec{x},\vec{y})+\beta f(\vec{x},\vec{z})$, para toda $\alpha,\beta\in\mathbb{R}$ y $\vec{x},\vec{y},\vec{z}\in V$. Un ejemplo de una forma bilineal es el producto punto en $V=\mathbb{R}^n$.

\subsubsection*{Clasificación de formas bilineales}

Sea $f:V\times V\to\mathbb{R}$ una forma bilineal. Esta es:

\begin{itemize}

\item \textbf{Simétrica:} Si para todo $\vec{x},\vec{y}\in V$ se tiene $f(\vec{x},\vec{y})=f(\vec{y},\vec{x})$.

\item \textbf{Antisimétrica:} Si para todo $\vec{x},\vec{y}\in V$ se tiene $f(\vec{x},\vec{y})=-f(\vec{y},\vec{x})$.

\item \textbf{Alternante:} Si para todo $\vec{x}\in V$ se tiene $f(\vec{x},\vec{x})=0$.

\end{itemize}

\subsubsection*{Bases en formas bilineales}

Si $f$ es una forma bilineal sobre $V$ y si $B=\{b_1,\ldots,b_n\}$ es una base para $V$, se pueden tomar $\vec{x},\vec{y}\in V$ arbitrarios tales que para todo $u,v\in V$, si $A=\big[a_{ij}\big]=\big[f(b_i,b_j)\big]$, entonces
$$f(u,v)=\big(\big[\vec{x}\big]_B\big)^T \cdot A \cdot \big(\big[\vec{y}\big]_B\big).$$

\subsection*{Formas cuadráticas}

Sea $V$ un espacio vectorial $n$-dimensional sobre $\mathbb{R}$. Sea $f$ una forma bilineal simétrica sobre $V$, y sea $A$ la matriz que representa a $f$ con respecto a la base canónica para $V$. El mapeo $q:V\to\mathbb{R}$ tal que $q(\vec{x})=f(\vec{x},\vec{x})=(\vec{x})^T\cdot A\cdot \vec{x}$ es llamado una forma cuadrática sobre $V$.

\subsubsection*{Clasificación de formas cuadráticas}

Sea $q:V\to\mathbb{R}$ una forma cuadrática. Esta es:

\begin{itemize}

\item \textbf{Positivo definida:} Si para todo $\vec{x}\neq 0$ en $V$ se tiene $q(\vec{x})>0$.

\item \textbf{Positivo semidefinida:} Si para todo $\vec{x}\neq 0$ en $V$ se tiene $q(\vec{x})\geq 0$.

\item \textbf{Negativo definida:} Si para todo $\vec{x}\neq 0$ en $V$ se tiene $q(\vec{x})<0$.

\item \textbf{Negativo semidefinida:} Si para todo $\vec{x}\neq 0$ en $V$ se tiene $q(\vec{x})\leq 0$.

\end{itemize}

\vspace{10pt}
\textbf{Nota: Aquí terminó la clase del 2 de noviembre del 2021.}




\vspace{10pt}
\textbf{Nota: Aquí terminó la clase del 4 de noviembre del 2021.}

\end{document}